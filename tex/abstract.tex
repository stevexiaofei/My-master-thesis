%# -*- coding: utf-8-unix -*-
% !TEX program = xelatex
% !TEX root = ../thesis.tex
% !TEX encoding = UTF-8 Unicode
%%==================================================
%% abstract.tex for SJTU Master Thesis
%%==================================================

\begin{abstract}
	目前,环境中化学品的数量日益增多,对人们的身体健康造成严重威胁,如何科学的对这些化学品进行风险评估
	是公共健康领域的热点与难点。传统的动物实验虽然研究结果真实可靠,但具有周期长、成本高以及涉及动物保护等局限性,
	难以进行高通量和大规模的毒性研究;使用细胞或者类器官等虽可进行较高通量的毒性评价与筛选,
	但其研究结果无法代替动物体内实验。秀丽隐杆线虫具有体积小、生命周期短、结构简单和高基因保守性等特点,
	是环境暴露和毒理学研究领域中重要的体内研究工具。微流控体系与线虫大小尺度相匹配,
	与传统的在琼脂板上进行的线虫实验相比,其具有反应体系小、通量高、自动化且操作灵活等优势。
	因此基于微流控芯片的线虫研究平台为高通量、大规模的环境化学品评估提供了新的手段。然而目前仍缺乏集成自动化操控
	及图像分析的一体化微流控线虫分析平台。
	
	针对上述问题,本文首先搭建了一个基于微流控芯片的硬件平台,可以在
	芯片上进行线虫的简单培养以及为线虫在给定环境毒素下的暴露创造一个精确的微环境。
	在自动化图像分析线虫的毒理实验中,体长和摆动频率往往作为评价毒性的重要生理指标,
	然而图像分析存在下述难点:\begin{enumerate*}[label=\arabic*. ]
	\item 由于线虫通体透明且由 PDMS 制作的微流控芯片也是透明的,
	因此线虫的前景轮廓分割是整个系统中的难点。\quad
	\item 当一个腔室中存在多个线虫时,
	多个线虫轮廓可能会纠缠在一起,由于无法辨识单个线虫的轮廓,从而导致跟踪丢失。\quad
	\item 由于线虫是非刚体其形态变化多种多样,线虫在多帧图像中的跟踪也是一个难点。\quad
	\end{enumerate*}
	针对上述问题,本论文提出如下解决方案:\begin{enumerate*}[label=\arabic*. ]
	\item 针对传统的图像分割算法在线虫前景轮廓分割任务中
	存在的不足 (如:面临鲁棒性差、依赖超参数选择和分割效果不理想等问题)。
	本文提出了一种基于深度卷积网络和条件随机场的分割算法。通过与其他分割算法对比, 
	表明本文提出的算法能够显著地改善线虫前景轮廓分割的性能。在我们标注的线虫前景分割数据集上,
	本文提出的分割算法在像素误差指标上达到 0.11\% 的分割精度。\quad
	\item 为了解决多线虫轮廓之间的纠缠,本文提出了一种基于卷积神经网络的编解码器架构,
		该模型有效的解决了单个线虫轮廓的解析问题。\quad
	\item 针对线虫的跟踪问题,本文提出了一种简单有效的跟踪策略,通过最近邻搜索匹配的方式实现线虫的跟踪。实验发现,
	在线虫轮廓分割较为完整的情况下,该跟踪方法能够对线虫轮廓实现鲁棒地跟踪。\quad
	\end{enumerate*}
	利用跟踪的线虫轮廓,可以计算出线虫的体长、摆动频率等生理特征。
	最后,通过该自动化平台研究了线性浓度梯度双氧水对线虫活性地影响,实验结果和传统的实验结果一致,
	展示了本文提出的基于微流控平台和自动化图像分析系统在毒理学研究中的应用前景。

	
\keywords{秀丽隐杆线虫; 微流控; 药物筛选; 轮廓分割; 卷积网络}
\end{abstract}
	
\begin{englishabstract}
	Currently, the number of chemicals in the environment is increasing and posing a serious threat to people's health. 
	How to conduct risk assessment of those chemicals scientifically is a hotspot and a difficult point in the public health field.
	Although traditional animal experiments are reliable and have great significance, 
	they have long cycle times, high costs, and limitations related to animal protection, 
	making it difficult to conduct high-throughput and large-scale toxicity studies.
	Cells or organs can be used  in  high throughput toxicity evaluation and screening,
	but the results of the study cannot replace experiment in vivo.
	The representative model of  Caenorhabditis elegans(C.elegans) has the characteristics of small size,
	short life cycle, simple structure and high gene conservation. 
	It is an important in-vivo research tool in the field of environmental exposure and toxicology research.
	The microfluidic system is matched with the size of the C.elegans. 
	Compared with the C.elegans experiment on the agar plate, 
	it has the advantages of small reaction system, high throughput, automation and flexible operation.
	Therefore, the microfluidic chip-based C.elegans research platform provides a new means for high-throughput,
	large-scale environmental chemical assessment. However, 
	there is still no integrated microfluidic analysis platform that
	integrates automated opration and image analysis.

	In response to the above problems,
	this paper first builds a hardware platform based on microfluidic chip, 
	which can realize the simple cultivation of C.elegans and create 
	a precise microenvironment for the exposure of C.elegans to a
	given environmental toxin.
	In the automated image analysis of toxicology experiments with C.elegans, 
	body length and swing frequency are often used as important physiological indicators
	for evaluating toxicity. However, image analysis has the following difficulties:
	\begin{enumerate*}[label=\itshape\alph*)\upshape]\quad
	\item Since the C.elegans is transparent and the microfluidic chip made by PDMS is also transparent, 
	the foreground contour segmentation of the C.elegans is a difficult point in the whole system.\quad
	\item When multiple C.elegans are present in one chamber, multiple contours of C.elegans may be
	entangled, resulting in loss of tracking due to the inability to recognize the contours of individual C.elegans.\quad
	\item Because C.elegans are non-rigid and their morphological changes are diverse, 
	it is also a difficult point to track the C.elegans in multi-frame images.\quad
	\end{enumerate*}
	To alleviate the problems mentioned above, this paper proposes the following solutions:
	\begin{enumerate*}[label=\itshape\alph*)\upshape]\quad
	\item Aiming at the shortcomings of the traditional image segmentation algorithm 
	in the foreground segmentation task of the worm (such as: poor robustness, 
	dependence on hyperparameter selection and unsatisfactory segmentation), this paper proposes a segmentation algorithm 
	based on deep convolutional networks and conditional random fields. Compared with other segmentation algorithms, 
	the proposed algorithm can significantly improve the performance of  foreground segmentation of C.elegans.
	On the segmentation dataset of the C.elegans we have labeled, 
	the segmentation algorithm proposed in this paper achieves 
	a segmentation accuracy of 0.11\% on the pixel error index.\quad
	\item In order to solve the entanglement between the contours of multi-worms, 
	this paper proposes a encode-decode architecture based on convolutional neural network, 
	which effectively solves the problem of parsing the contour of a single C.elegans.\quad
	\item In order to achieve multi-worms tracking, this paper proposes a simple and effective 
	tracking strategy to track the C.elegans by nearest neighbor search.
 The experiment found that the tracking method can achieve robust
 tracking of the C.elegans under the condition that 
 the foreground segmentation of the C.elegans is complete.\quad
\end{enumerate*}
 Using the traced contour of C.elegans, physiological characteristics such as 
 body length and swing frequency of the C.elegans can be calculated.
 Finally, by studying the effect of linear concentration gradient
 hydrogen peroxide on C.elegans activity and the results are consistent with the
traditional experiment results. The application prospects
 of microfluidic platform and automated image analysis system in toxicology research are presented.
 
\englishkeywords{\large C. elegans, Microfludics, Drug screening, Contour segmentation, Convolutional network}
\end{englishabstract}

