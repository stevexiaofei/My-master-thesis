%# -*- coding: utf-8-unix -*-
% !TEX program = xelatex
% !TEX root = ../thesis.tex
% !TEX encoding = UTF-8 Unicode
%%==================================================
%% abstract.tex for SJTU Master Thesis
%%==================================================

\begin{abstract}
	目前,环境中化学品的数量日益增多且对人们的身体健康造成严重威胁,如何科学的进行风险评估是公共健康领域的热点
	与难点。使用细胞或者类器官可进行较高通量的毒性评价与筛选,但研究结果无法代替动物体
	内实验。传统的动物实验虽然研究结果意义较大,但具有周期长、成本高和涉及动物保护等局限性,难以进行高通量的毒性研究。
	代表性模式生物秀丽隐杆线虫具有体积小、生命周期短、结构简单和高基因保守性等特点,
	是环境暴露和毒理学研究领域中重要的体内研究工具。微流控体系与线虫大小尺度相匹配,
	与在琼脂板上进行的线虫实验相比,其具有反应体系小、通量高、自动化且操作灵活等优势。
	在线虫的毒理实验中,体长和摆动频率往往作为评价毒性的重要生理指标,通过人工的观察
	不仅通量低而且还会引入误差。
	
	针对上述问题,本文首先搭建了一个基于微流控芯片的硬件平台,可以实现线虫
	的简单培养以及为线虫在给定环境毒素下的暴露创造一个精确的微环境,由于线虫通体透明且由
	PDMS制作的微流控芯片也是透明的,线虫的分割是整个系统中的难点,本文首先采用了背景减除的方法进行
	线虫前景轮廓的提取。实验发现,该方法在较低背景噪声的情况下可以较好的分割线虫。但面对复杂的
	背景时,该方法的鲁棒性出现显著的下降,导致后续线虫跟踪的丢失率显著增加。针对背景减除方法的不足,
	本文提出了一种基于深度卷积网络的分割模型。通过实验对比,在我们标注的数据集上分割误差下降了xx。另外,
	线虫的跟踪也是一个难点,由于线虫是非刚体其形态变化多种多样。为此,本文提出了一种简单有效的跟踪策略,
	能过最近邻搜索匹配的方式实现线虫的跟踪。实验发现,在线虫轮廓分割较为完整的情况下,该跟踪方法
	能够对线虫轮廓实现鲁棒的跟踪。利用跟踪的线虫轮廓,可以计算出线虫的体长、摆动频率和运动速度等生理特征。
	最后,通过线虫和双氧水的相互作用展示了本文提出的基于微流控平台和自动化图像分析系统在毒理学研究中的
	应用前景。
	
\keywords{秀丽隐杆线虫; 微流控; 毒性评价; 生理监测}
\end{abstract}

\begin{englishabstract}

	Todo

\englishkeywords{\large C. elegans, Microfludics, Toxicity evaluation, Physiological monitoring}
\end{englishabstract}

