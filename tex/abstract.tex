%# -*- coding: utf-8-unix -*-
% !TEX program = xelatex
% !TEX root = ../thesis.tex
% !TEX encoding = UTF-8 Unicode
%%==================================================
%% abstract.tex for SJTU Master Thesis
%%==================================================

\begin{abstract}
	秀丽隐杆线虫具有体积小、生命周期短、结构简单和基因保守性高等特点,
	是环境暴露和毒理学研究领域中重要的活体研究工具。微流控体系与线虫大小尺度相匹配,
	与传统的在琼脂板上进行的线虫实验相比,其具有反应体系小、通量高、自动化且操作灵活等优势。
	因此基于微流控芯片的线虫研究平台为高通量、大规模的环境化学品评估提供了新的手段。然而目前仍缺乏集成自动化操控
	及图像分析的一体化微流控线虫分析平台。
	
	针对上述问题,本文首先设计了两个分别用于急性毒理实验的双层阀门梯度芯片以及用于长期观察的单层阀门芯片。
	搭建了集成流体进样、振荡及阀门开关的流体控制系统,通过多线程优化了线虫视频采集的速度。
	为线虫在给定环境毒素下的暴露创造一个精确的微环境及控制观察系统。
	针对线虫的视频特征提取存在的难点,本论文提出:\begin{enumerate*}[label=\arabic*. ]
	\item  针对传统的图像分割算法在线虫前景轮廓分割任务中存在的不足 (如:
	面临鲁棒性差、依赖超参数选择和分割效果不理想等问题)。
	提出了一种基于深度卷积网络和条件随机场的分割算法。通过与其他分割算法对比,
	表明本文提出的算法能够显著地改善线虫前景轮廓分割的性能。\quad
	\item 为了解决多线虫轮廓之间的纠缠,本文提出了一种基于卷积神经网络的编解码器架构,
	该模型有效地解决了单个线虫轮廓的解析问题。\quad
	\item 针对线虫的跟踪问题,本文提出了一种简单有效的跟踪策略,
	通过最近邻搜索匹配的方式实现线虫的跟踪。
	该跟踪方法能够对线虫轮廓实现鲁棒的跟踪。\quad
	\end{enumerate*}
利用上述算法,可以自动计算出线虫的体长、摆动频率等生理特征。
最后,本文分别针对两款线虫芯片及相关软硬件平台开展了相关实验,
其中急性线虫毒理芯片可自动化研究不同浓度梯度的双氧水溶液对秀丽隐杆线虫摆动频率的影响,
实验结果与传统实验结果一致。慢性毒理芯片显示该芯片可以控制腔室中线虫的数量,
通过开发的算法可进行自动视频线虫追踪及线虫特征分析。展示了本文提出的
基于微流控平台和自动化图像分析系统在毒理学研究中的应用前景。
	
\keywords{秀丽隐杆线虫; 微流控; 毒理测试; 特征提取; 卷积网络}
\end{abstract}
	
\begin{englishabstract}
	The representative model of  Caenorhabditis elegans(C.elegans) has the characteristics of small size,
	short life cycle, simple structure and high gene conservation. 
	It is an important in-vivo research tool in the field of environmental exposure and toxicology research.
	The microfluidic system is matched with the size of the C.elegans. 
	Compared with the C.elegans experiment on the agar plate, 
	it has the advantages of small reaction system, high throughput, automation and flexible operation.
	Therefore, the microfluidic chip-based C.elegans research platform provides a new means for high-throughput,
	large-scale environmental chemical assessment. However, 
	there is still no integrated microfluidic analysis platform that
	integrates automated opration and image analysis.

	In response to the above problems, a hardware system platform based on microfluidic chip 
	is set up firstly, including the design and fabrication of two microfluidic chips and 
	the optimization of the speed of  video acquisition.
	These two chips provide a precise microenvironment for the exposure of C.elegans
	to a given environmental toxin. 
	In the toxicological experiment based on C.elegans, the body length and 
	swing frequency are often used as important physiological indexes to evaluate 
	toxicity.
	However, by using computer vision to extract the features of C.elegans, 
	the following difficulties exist:
	\begin{enumerate*}[label=\itshape\alph*)\upshape]\quad
	\item Since the C.elegans is transparent and the microfluidic chip made by PDMS is also transparent, 
	the foreground contour segmentation of the C.elegans is a difficult point in the whole system.\quad
	\item When multiple C.elegans are present in one chamber, multiple contours of C.elegans may be
	entangled, resulting in loss of tracking due to the inability to recognize the contours of individual C.elegans.\quad
	\item Because C.elegans are non-rigid and their morphological changes are diverse, 
	it is also a difficult point to track the C.elegans in multi-frame images.\quad
	\end{enumerate*}
	To alleviate the problems mentioned above, this paper proposes the following solutions:
	\begin{enumerate*}[label=\itshape\alph*)\upshape]\quad
	\item Aiming at the shortcomings of the traditional image segmentation algorithm 
	in the foreground segmentation task of the worm (such as poor robustness, 
	dependence on hyperparameter selection and unsatisfactory segmentation), this paper proposes a segmentation algorithm 
	based on deep convolutional networks and conditional random fields. Compared with other segmentation algorithms, 
	the proposed algorithm can significantly improve the performance of  foreground segmentation of C.elegans.
	On the segmentation dataset of the C.elegans we have labeled, 
	the segmentation algorithm proposed in this paper achieves 
	a segmentation accuracy of 0.11\% on the pixel error index.\quad
	\item In order to solve the entanglement between the contours of multi-worms, 
	this paper proposes a encode-decode architecture based on convolutional neural network, 
	which effectively solves the problem of parsing the contour of a single C.elegans.\quad
	\item In order to achieve multi-worms tracking, this paper proposes a simple but effective 
	tracking strategy to track the C.elegans by nearest neighbor search.
 The experiment found that the tracking method can achieve robust
 tracking of the C.elegans under the condition that 
 the foreground segmentation of the C.elegans is complete.\quad
\end{enumerate*}
 Using the traced contour of C.elegans, physiological characteristics such as 
 body length and swing frequency of the C.elegans can be calculated.
Finally, the effect of linear concentration gradient hydrogen peroxide on C.elegans activity 
is studied by this automatic platform, and the experimental results 
are consistent with the traditional experimental results. 
The application prospects
 of microfluidic platform and automated image analysis system in toxicology research are presented.
 
\englishkeywords{\large C. elegans, Microfludics, Toxicological test, Feature extraction, Convolutional network}
\end{englishabstract}

