\chapter{微流控线虫芯片的设计及硬件平台的搭建}
\section{引言}
	在传统的药物筛选过程中,往往是通过人工的方式在96孔板上配置不同浓度的化合物。然后将线虫暴露在不同浓度的化合物下,
	观察并记录线虫在不同浓度化合物下的变化。这种人工稀释的方法不仅存在样品消耗大、通量低和操作繁琐等缺点,而且也不利于
	观察。近年来,随着微流控技术的发展,一些研究者们提出了基于微流控技术的片上浓度稀释的方法。片上浓度梯度稀释的方法
	不仅使反应体系减小十倍甚至百倍,而且极大的减少了试剂的消耗,节约了实验成本。虽然已经有“圣诞树”结构的被动式梯度
	形成芯片的报道,但存在样品消耗大以及需要精确的流阻设计和流速调节。为了研究多种化合物的复合对线虫活性的影响,
	本文设计了一种基于振荡的线性梯度稀释的微流控芯片。且只需要一个气源即可通过振荡的方式
	完整样品的快速稀释,并通过染料实验和荧光实验验证本文提出的线性浓度梯度芯片的可行性。
	
\section{材料与方法}
\subsection{实验材料与仪器}

\subsection{芯片加工工艺}
\subsection{线虫梯度芯片的设计}
\subsection{线虫梯度芯片的制作}
\section{芯片方案的验证}
\section{本章小结}
