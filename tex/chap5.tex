\chapter{基于深度卷积网络的线虫前景轮廓提取}
\section{引言}
秀丽隐杆线虫由于通体透明,且PDMS制作的微流控芯片也是透明的,所以线虫轮廓的分割是个难点。
通过简单的阈值分割的方法,并不能够很好的分割线虫的轮廓,因为前景像素的灰度值范围被包括在背景像素的灰度范围内。
通过背景减除的方法分割的线虫轮廓。由于图像噪声的影响,会导致分割的线虫轮廓出现断裂、不完整等情况。
这些都会导致线虫跟踪的失败。另一方面,背景的估计需要$50\sim100$帧背景不变的图像。
当需要对线虫进行实时跟踪时,线虫分割算法需要一个启动时间用于背景的估计。
另外,当CCD相机或载物台移动时(如:需要观察不同腔室中的虫子),由于背景的改变,这种算法还是会失效。
卷积网络作为一种强大的特征提取器在许多计算机视觉中取得了很大的成功。

	本课题将卷积网络用于线虫轮廓的前景提取具有如下优势:
	
\begin{enumerate}
  \item 与基于背景减除的方法相比,基于卷积网络的分割不需要对图像背景建模,只依赖当前帧的图像,从而能够保证实时性的要求。
  \item 能够降低硬件成本,传统的线虫图像处理,为了获得一个背景和线虫轮廓对比度比较高的图像。通常使用特制的硬件对CCD和照明都有很高的要求。卷积网络作为一种强大特征提取器降低了对图像质量的要求。
  \item 鲁棒性更好,传统的线虫轮廓分割算法,通常需要人工的选取一些超参数(如:分割的阈值,形态学操作中核的大小等等),但由于视频采集过程中照明的变化以及图像噪声的影响很难选取一个最佳的全局参数。
基于卷积网络的分割是一种端到端的方法,输出直接是分割的结果,因此,这种方法不依赖超参数的选取,具有很好的鲁棒性。
\end{enumerate}

\section{数据集准备}
\section{U-net卷积网络介绍}

\section{U-net卷积网络的改进}

\section{本章小结}