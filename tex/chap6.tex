\chapter{总结与展望}
\section{工作总结}
	目前环境中大量的化合物对人类的健康造成了很大的威胁,如何高效快速的评估这些化合物对人类
	健康的影响是目前毒理学研究面临的一个巨大挑战。秀丽隐杆线虫作为一种重要的模式生物,
	在现在毒理学测试发挥着重要作用。而传统的线虫实验通常在96孔板上完成,需要大量
	的人工操作,不仅操作复杂而且通量低。另一方面,在实验过程中,通过人工观察的方式对线虫相关生理特征
	(如:摆动频率、体长等)的统计,不仅效率低下,而且还会引入人为误差。这些都大大地限制了大规模
	毒理实验的研究进展。本文的工作为基于秀丽隐杆线虫的毒理学测试
	提供了一个集成微流控芯片和自动化视频特征提取的软硬件平台。本文的主要工作如下:
	\begin{enumerate}[label={(\arabic*)},font={\color{black!50!black}\bfseries}]
	\item 设计了一款线性浓度梯度稀释的双层微流控芯片,并介绍了双层微流控芯片的制作工艺
	(包括从芯片结构设计、芯片模具的制作到微流控芯片的制作),通过染料和荧光实验验证了线性
	梯度稀释芯片的合理性。
	\item 线性梯度稀释芯片虽然在自动化片上梯度形成方面具有优势,但由于没有设计食物供应的通道,
	因此并不适合线虫的长期培养观察,比较适合线虫的急性毒理实验。基于此,本文又设计了一款
	单层侧向阀门的线虫培养芯片。侧向阀门的设计可以控制腔室中线虫的数量,食物可以通过网状结构
	过滤后流向线虫腔室,为线虫提供食物,虫卵可以通过侧向的管道排出线虫培养腔室。从而可以对
	同一代的线虫进行长期地培养观察。
	\item 基于相机厂商提供的SDK,开发了一个高速的线虫视频采集程序。首先介绍了相机设备的操作流程,
	详细分析了从图像采集到视频写入过程中的时间开销,并基于时间开销的分析结果,提出了一种多线程
	的方法将图像采集和视频写入这两个任务并行。与单线程的视频采集相比,多线程的方式能够显著提高
	视频采集的帧率。最后测试了不同采集分辨率对视频帧率的影响。
	\item 针对线虫视频特征提取任务的特点,本文采用了“前景轮廓分割——轮廓解析
——轮廓跟踪——特征提取”的技术路线。针对传统的图像处理方法在线虫前景轮廓分割任务中存在的不足
(如:鲁棒性不足、线虫轮廓出现断裂和依赖超参数的选择等),本文提出了一种基于条件随机场的
深度卷积分割算法,通过与传统的图像分割方法相比,本文提出的前景分割算法显著地改善了线虫前景轮廓
分割的性能。
	\item 针对多线虫轮廓跟踪过程中,多线虫轮廓相互纠缠导致无法区分单个线虫的轮廓,
	从而导致线虫轮廓丢失的问题本。文提出了一种设计了一个基于深度卷积的SingleOut-Net网络,可以成功解析出单个
	线虫的轮廓,并比较了不同的网络架构对网络性能、模型复杂度和实时性的影响。
	\item 基于线虫轮廓跟踪的结果,本文提出了一种简单有效的线虫轮廓跟踪算法成功实现了对线虫的鲁棒跟踪。该方法
	首先找出相邻两帧图像中所有线虫轮廓的重心,然后通过最近邻匹配的方式找出相邻两帧图像中线虫轮廓之间的对应关系,
	从而实现对线虫轮廓的跟踪。
	\item 基于线虫轮廓跟踪的结果,本文介绍了线虫轮廓中间脊线的提取方法,以及基于线虫的中间脊线进行线虫摆动
	频率的计算。
	\end{enumerate}
	
\section{展望}
	本文的主要工作是为基于秀丽隐杆线虫的大规模毒理学测试和药物筛选搭建一个软硬件平台。








