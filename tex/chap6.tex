\chapter{总结与展望}
\section{工作总结}
	目前环境中大量的化合物对人类的健康造成了很大的威胁,如何高效快速地评估这些化合物对人类
	健康的影响是目前毒理学研究面临的一个巨大挑战。秀丽隐杆线虫作为一种重要的模式生物,
	在现在毒理学测试中发挥着重要作用。而传统的线虫实验通常在96孔板上完成,需要大量
	的人工操作,不仅操作复杂而且通量低。另一方面,在实验过程中,通过人工观察的方式对线虫相关生理特征
	(如:摆动频率、体长等)的统计,不仅效率低下,而且还会引入人为误差。这些都大大地限制了大规模
	毒理实验的研究进展。本文的工作为基于秀丽隐杆线虫的毒理学测试
	提供了一个集成微流控芯片和自动化视频特征提取的软硬件平台。本文的主要工作如下:
	\begin{enumerate}[label={(\arabic*)},font={\color{black!50!black}\bfseries}]
	\item 本文设计了两款线虫微流控芯片,分别用于线虫急性毒理实验和可以控制线虫
	进样的长期培养芯片。针对用于急性毒理实验的线性浓度梯度稀释芯片,介绍了芯片
	结构设计和双层微流控芯片的制作工艺 (包括芯片模具的制作、微流控芯片的制作等
	工艺步骤),通过染料和荧光实验验证了线性梯度稀释芯片设计的合理性。线性梯度稀释芯
	片虽然在自动化片上梯度形成方面具有优势,但由于没有设计食物供应的通道,因此并
	不适合线虫的长期培养观察,比较适合线虫的急性毒理实验。基于此,本文又设计了一款
	单层带侧向阀门的线虫培养芯片。侧向阀门的设计可以控制腔室中线虫的数量,食物可以
	通过网状结构过滤后流向线虫腔室,为线虫提供食物,虫卵可以通过侧向的管道排出线虫
	培养腔室。从而可以对同一代的线虫进行长期地培养观察。
	\item 基于相机厂商提供的 SDK,开发了一个高速的线虫视频采集程序。
	由于相机的SDK是以动态链接库的形式提供的,而本文将python作为开发语言。
	为了调用相机的API,本文用python的ctypes库对动态链接库中的函数进行了封装以供
	python调用。通过分析了从图像采集到视频写入过
	程中的时间开销,提出了一种多线程的方法将图像采集和视频写入这两个任务并行。
	与单线程的视频采集相比,多线程的方式能够显著提高视频采集的帧率。
	最后测试了不同采集分辨率对视频帧率的影响。
	\item 针对线虫视频特征提取任务的特点,本文采用了“前景轮廓分割——轮廓解析——轮廓跟踪——特征提取”
	的技术路线。针对传统的图像处理方法在线虫前景轮廓分割任务中存在的不足 
	(如:鲁棒性不足、线虫轮廓出现断裂和依赖超参数的选择等),
	本文提出了一种基于条件随机场的深度卷积分割算法,通过与传统的图像分割方法相比,
	本文提出的前景分割算法显著地改善了线虫前景轮廓分割的性能。
针对多线虫轮廓跟踪过程中,多线虫轮廓相互纠缠导致无法区分单个线虫的轮廓,
从而引起线虫轮廓丢失的问题。本文设计了一个基于深度卷积的 
SingleOut-Net 网络,可以成功解析出单个线虫的轮廓,并比较了不同的网络架构
在网络性能、模型复杂度和实时性方面的差异。

	\item 基于线虫轮廓跟踪的结果,本文提出了一种简单有效的线虫轮廓跟踪算法,
	成功实现了对线虫的鲁棒跟踪。该方法首先找出相邻两帧图像中所有线虫轮廓的重心,
	然后通过最近邻匹配的方式找出相邻两帧图像中线虫轮廓之间的对应关系,
	从而实现对线虫轮廓的跟踪。基于线虫轮廓跟踪的结果,本文介绍了线虫轮廓
	中间脊线的提取方法,以及基于线虫的中间脊线介绍了线虫摆动频率和体长的计算方法。
	\end{enumerate}
	
\section{展望}
	本课题拟为基于秀丽隐杆线虫的大规模毒理学测试和药物筛选搭建一个软硬件平台。
	在芯片操作方面,该平台可以实现线虫的长期培养、自动给药、片上梯度形成等
	自动化操作。在生理特征检测方面,该平台可以实现体长、身体弯曲频率、头部
	摆动频率、产卵率等生理特征的监测。但限于时间关系,本文的工作并没有将
	这些功能全部包括。结合本文已经完成的工作,未来该平台可以在以下几个方面进一步优化:
	
	\begin{enumerate}[label={(\arabic*)},font={\color{black!50!black}\bfseries}]
	\item 线虫产卵率的监测对于生殖毒性的评价而言,是非常重要的指标。
	下一阶段可以考虑在芯片设计和虫卵识别计数两个方面开展工作。在芯
	片设计方面,可以设计一款具备将线虫和虫卵分离同时将虫卵收集在一个
	腔室功能的芯片,并通过图像识别算法对这个腔室中的虫卵进行计数。
	\item 在神经发育毒性的评价中,头部摆动频率也是一项十分重要的指标。
	现阶段本平台已经具备了对线虫进行前景轮廓分割、轮廓解析、轮廓跟踪、
	线虫轮廓中间脊线提取和线虫身体弯曲频率计算等功能。下一阶段,
	可以基于目前线虫轮廓中间脊线的提取结果完成头部摆动频率的计算。
	\item 在芯片设计上,充分利用微流控芯片高通量的优势。
	下一阶段的工作可以考虑设计具备形成多种化合物复合浓度梯度功能的线虫芯片,
	同时该芯片上应包含更多的线虫培养腔室。进一步体现该平台在复合药物毒性评价方面
	高通量的优势。
	\item 为了探索毒性标志,未来可以将线虫微流控芯片和质谱仪连接,对线虫的代谢产物进行毒性鉴别。

	\end{enumerate}








